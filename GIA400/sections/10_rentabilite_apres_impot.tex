\section{Analyse de rentabilité de projets après impôt}
\subsection{Qu’est-ce qu’un bien amortissable?}
\begin{enumerate}
    \item Un bien utilisé dans le cadre d’activités économiques ou détenu pour produire des bénéfices. • Principe fiscal important : une activité ou un amortissement sur un bien n’ayant aucune chance de produire des bénéfices n’est pas déductible aux fins d’impôt.
    \item Un bien ayant une vie utile définissable et supérieure à une année.
    \item Un bien qui s’use, se détériore, perdant ainsi de la valeur. • Amortissable : équipements, matériel roulant, bâtiments. • Non amortissable : terrains.
\end{enumerate}


\subsection{Coût amortissable}
\subsubsection{Principes comptables}
- Toutes les dépenses engagées pour acquérir et mettre en service sont amortissables.

- L’amortissement commence seulement lorsque le bien entre effectivement en service.


\subsection{Amortissement comptable}
- « Rapprocher » le mieux possible le coût d’un bien des revenus générés par ce bien.

- Utilisé dans les états financiers publiés dans les rapports annuels, les états financiers internes.

\subsection{Amortissement fiscal}
- Amortissement permis par l’agence du revenu du Canada (ARC) et Revenu Québec en vue du calcul de la déduction pour amortissement (DPA).

- En général, permet un amortissement plus rapide que l’amortissement comptable au début de la vie d’un bien. Ceci encourage l’investissement par les entreprises.

- Le gain fiscal ainsi réalisé (qui est en fait une subvention gouvernementale) est rapporté à la rubrique « Impôts reportés » au bilan.