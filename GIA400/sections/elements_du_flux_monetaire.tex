\section{Évaluation des flux monétaires des projets}
\pRef{Chapitre 9}

\subsection{Éléments des flux monétaires}
\pRef{P.570}

- Nouveaux investissements et immobilisations existantes \\
- Valeur de récupération (ou prix de vente net) \\
- Investissements dans le fonds de roulement \\
- Libération du fonds de roulement \\
- Revenus ayant un effet sur la trésorerie/économies\\
- Coûts d'exploitation\\
- Charge de location\\
- Intérêts et remboursement des sommes empruntées\\
- Impôt sur le revenu et les crédits d'impôts

\subsection{Sorties de fonds}
- Investissement initial\\
- Investissement dans le fond de roulement\\
- Coûts d'exploitation\\
- Versements d'intérêts et de capital sur un emprunt\\
- Impôt sur le revenu\\

\subsection{Rentrées de fonds}
- Revenus différentiels\\
- Réduction des coûts (économies sur les coûts)\\
- Crédits d'impôts autorisés\\
- Valeur de récupération\\
- Libération de l'investissement dans le fonds de roulement\\
- Montants provenant des emprunts à court et long terme

\subsection{Activités d'exploitation}
\pRef{P.574}
Ces activités comprennent les fonds nets provenant de l'exploitation de l'entreprise, c'est-à-dire le bénéfice net (ou la perte nette) auxquels l'on ajoute ou retranche les éléments hors-fonds. Ces éléments sont des produits ou des charges qui ne représentent pas une augmentation ou une diminution du fonds de roulement.

\subsection{Activités d'investissement}
\pRef{P.575}
Les activités de financement comprennent les variations des liquidités et tiennent compte de toutes les sources de financement de l'entreprise, qu'il s'agisse d'emprunts à long terme ou d'émission de capital-actions.

Il ne faut pas oublier qu'il faut y présenter toutes les activités de financement, qu'elles soient positives ou négatives.

{\scriptsize
*On entend par positives les émissions d'actions et les emprunts à long terme; par négatives, les remboursements d'emprunts et les rachats d'actions.\\
**Une particularité est à noter en ce qui concerne le paiement des dividendes. Certains le tiennent pour une activité de financement et d'autres pour une opération liée à l'exploitation courante.
}

\subsection{Activités de financement}
\pRef{P.575}
Les activités d'investissement, on tient compte de tous les investissements à long terme effectués par l'entreprise comme :\\
- Les placements à long terme\\
- L'acquisition d'immobilisations corporelles et incorporelles\\
Lorsque l'entreprise vend ces mêmes actifs, les sommes reçues sont alors considérées comme des diminutions des activités de financement.
