\begin{tikzpicture}
\node [mybox] (box){%
    \begin{minipage}{0.3\textwidth}
    % \txt{\footnotesize
    %     Il existe une paire de points $f_o$ et $f_i$ ayant les propriétés: 1) Un obj placé en $f_o$ donne des rayons réfléchis parallèles. 2) Des rayons incidents parallèles convergeront en $f_i$
    % }
    $$\frac{1}{f}=\frac{1}{s_o}+\frac{1}{s_i}=\frac{2}{R}$$
    \txt{
    \begin{center}
        \textbf{Si} $s_i=\infty$, \textbf{alors} $s_o=f_o=\frac{R}{2}$,\hskip15pt
        \textbf{Si} $s_o=\infty$, \textbf{alors} $s_i=f_i=\frac{R}{2}$\\
        Donc $f_i=f_o=f=\frac{R}{2}$
    \end{center}
        \textbf{Miroir concave:} Si $R > 0$, alors $f > 0$, 
        \textbf{Miroir convexe:} Si $R < 0$, alors $f < 0$,
        \textbf{Miroir plan:} $R=\infty$ donc $s_i=-s_o$ et $g_t=\frac{-s_i}{s_o}=+1$
    }
    \end{minipage}
};
\node[fancytitle, right=10pt] at (box.north west) {Miroirs};
\end{tikzpicture}