\begin{tikzpicture}
\node [mybox] (box){%
    \begin{minipage}{0.3\textwidth}
    {\footnotesize
    \textbf{Définition:} Lentille considérée comme épaisse lorsque la distance $a$ entre les sommets $V_1$ et $V_2$ des surfaces n'est plus négligeable devant les rayons de courbure des surfaces sphériques. $V_1$ et $V_2$ sont les points gauche et droite de la lentille. $a$ est la distance entre $V_1$ et $V_2$, donc largeur de la lentille.}
    \txt{
        $$\frac{1}{f}\ =\ \frac{1}{s_o}+\frac{0}{s_i}\ =\ (n_{la}-1)[\frac{1}{R_1}-\frac{1}{R_2}+\frac{(n_{la}-1)\ a}{n_{la}\ R_1\ R_2}]$$
        {\footnotesize *où $n_{la}=\frac{n_l}{n_a}$\\}
        
        \textbf{Plans principaux}\\
        {\footnotesize$\bar{V_1 H_1}$ et $\bar{V_2 H_2}$ > 0 lorsque le déplacement de V vers H se fait dans le sens de la propagation de la lumière.}
        $$\bar{V_1 H_1}=\frac{-f(n_{la}-1)\ a}{R_2\ n_{la}}$$
        $$\bar{V_2 H_2} =\ \frac{-f(n_{la}-1)\ a}{R_1\ n_{la}} =\ \bar{V_1 H_1}(\frac{R_2}{R_1})$$
	}
	{\footnotesize
	- $s_o$ et $f_o$ mesurés avec $H_1$, donc $s_o=distance+\bar{V_1H_1}$\\
	- $s_i$ et $f_i$ mesurés avec $H_2$, donc $s_i=s_i+\bar{V_2H_2}$\\ ($solve(\frac{1}{f}=\frac{1}{s_o}+\frac{1}{s_i},s_i)$ puis $s_i=s_i+\bar{V_2 H_2}$)}
    \end{minipage}
};
\node[fancytitle, right=10pt] at (box.north west) {Lentilles épaisses};
\end{tikzpicture}